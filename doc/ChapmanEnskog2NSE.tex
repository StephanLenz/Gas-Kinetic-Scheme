\documentclass[
	pdftex,             % Ausgabe des Latex-Dokuments als PDF
	12pt,				% Schriftgroesse 12pt
	a4paper,		   	% Seiten Layout
	english,				% Sprache, global
	oneside,			% Einseitiger Druck
]{article}

% Seitenränder
\usepackage{geometry}
\geometry{a4paper, top=30mm, left=15mm, right=15mm, bottom=25mm,
headsep=15mm}

\usepackage[T1]{fontenc}
\usepackage[utf8]{inputenc}
\usepackage[english]{babel}

% Erweiterte Mathematikbibliotheken
\usepackage[fleqn]{amsmath}
\usepackage{amssymb}

% Zum einbinden von Grafiken  
\usepackage{graphicx}
\usepackage{epstopdf}
\usepackage{wrapfig}

% Benutzerdefinierte Kopf- und Fußzeile
\usepackage{scrpage2}
\pagestyle{scrheadings}
\setheadsepline{1pt} % Linie unter dem Header
\ihead{Gas-Kinetic-Scheme \\ Documentation}
\ohead{Stephan Lenz \\ IRMB -- TU Braunschweig}
\ofoot{\pagemark}

% Einr�cken nach Absatz verhinder
\setlength{\parindent}{0pt}

\usepackage{color}

% =====================================================

\newcommand{\mom}[1]{\langle #1 \rangle}
\newcommand{\uu}[1]{\underline{#1}}
\newcommand{\uuu}[1]{\underline{\underline{#1}}}
\newcommand{\vv}[1]{\vec{#1}}
\DeclareMathOperator{\sign}{sign}
\newcommand{\dxi}{~\textnormal{d}\Xi}
\newcommand{\pdxi}{~\uu{\psi}~\textnormal{d}\Xi}
\newcommand{\pp}[1]{\dfrac{\partial}{\partial #1}}
\newcommand{\ppt}{\pp{t}}
\newcommand{\ppx}{\pp{x}}
\newcommand{\ppy}{\pp{y}}

% =====================================================

\begin{document}

\section{With Chapman-Enskog-Expansion from BGK to NSE}

The first part is still not clear.

\begin{equation}
f = g - \epsilon \hat{\tau} \textnormal{D}_{\vv{u}} g 
      + \epsilon^2 \hat{\tau} \textnormal{D}_{\vv{u}} \left( \hat{\tau} \textnormal{D}_{\vv{u}} g \right) 
      + \mathcal{O} \left( \epsilon^2 \right)
\end{equation}

Compatibility Condition:

\begin{equation}
\int (f-g) \pdxi
=
\epsilon \int \hat{\tau} \textnormal{D}_{\vv{u}} g \pdxi
-
\epsilon^2 \int \hat{\tau} \textnormal{D}_{\vv{u}} \left( \hat{\tau} \textnormal{D}_{\vv{u}} g \right) \pdxi
= 0
\end{equation}

With $\epsilon$ and $\hat{\tau}$ independent on $u$, $v$ and $\xi$:

\begin{equation}
\underbrace{
    \int \textnormal{D}_{\vv{u}} g \pdxi
}_{\uu{L}}
=
\underbrace{
    \epsilon \int \textnormal{D}_{\vv{u}} \left( \hat{\tau} \textnormal{D}_{\vv{u}} g \right) \pdxi
}_{\epsilon \uu{R}}
\end{equation}

\begin{equation}
\uu{L} = \ppt \int g \pdxi + \ppx \int u g \pdxi + \ppy \int v g \pdxi
\end{equation}

\begin{equation}
\begin{array}{lcl}
L_0 &=& \ppt \rho \mom{1} + \ppx \rho \mom{u} + \ppy \rho \mom{v} \\
    &=& \ppt \rho + \ppx \left( \rho U \right) + \ppy \left( \rho V \right)
\end{array}
\end{equation}

\begin{equation}
\begin{array}{lcl}
L_1 &=& \ppt \left( \rho \mom{u} \right)
      + \ppx \left( \rho \mom{u^2} \right)
      + \ppy \left( \rho \mom{u} \mom{v} \right) \\
    &=& \ppt \left( \rho U \right)
      + \ppx \left( \rho U + p \right)
      + \ppy \left( \rho U V \right)
\end{array}
\end{equation}

\begin{equation}
\begin{array}{lcl}
L_2 &=& \ppt \left( \rho \mom{v} \right)
      + \ppx \left( \rho \mom{u} \mom{v} \right)
      + \ppy \left( \rho \mom{v^2} \right) \\
    &=& \ppt \left( \rho V \right)
      + \ppx \left( \rho U V \right)
      + \ppy \left( \rho U + p \right)
\end{array}
\end{equation}

\begin{equation}
\begin{array}{lcl}
L_3 &=& \ppt \rho \mom{\frac{1}{2} (u^2 + v^2 + \xi^2)} 
      + \ppx \rho \mom{\frac{1}{2} (u^3 + uv^2 + u\xi^2)}
      + \ppy \rho \mom{\frac{1}{2} (u^2v + v^3 + v\xi^2)}
\end{array}
\end{equation}

\begin{equation}
\begin{array}{lcl}
\ppx \rho \mom{\frac{1}{2} (u^3 + uv^2 + u\xi^2)}
    &=& \ppx \rho \left( \frac{1}{2}\mom{u^3} + \frac{1}{2}\mom{uv^2} + \frac{1}{2}\mom{u\xi} \right)
 \\ &=& \ppx \rho \left( \dfrac{1}{2} \left( U \left( U + \dfrac{1}{2\lambda} \right) + \dfrac{2}{2\lambda}U \right)
                       + \dfrac{1}{2} U \left( V^2 + \dfrac{1}{2\lambda} \right)
                       + \dfrac{1}{2} U \dfrac{K}{2\lambda} \right)
 \\ &=& \ppx U \left( \dfrac{2+K}{4\lambda}\rho + \dfrac{1}{2} \rho \left( U^2 + V^2 \right) + \dfrac{\rho}{2\lambda} \right)
 \\ &=& \ppx U \left( \rho E + p \right)
\end{array}
\end{equation}


\begin{equation}
\begin{array}{lcl}
L_3 &=& \ppt \left( \rho E \right)
      + \ppx U \left( \rho E + p \right)
      + \ppy V \left( \rho E + p \right)
\end{array}
\end{equation}

\begin{equation}
\uu{L}
=
\ppt
\begin{pmatrix}
\rho \\ \rho U \\ \rho V \\ \rho E
\end{pmatrix}
+
\ppx
\begin{pmatrix}
\rho U \\ \rho U^2 + p \\ \rho U V \\ U \left( \rho E + p \right)
\end{pmatrix}
+
\ppy
\begin{pmatrix}
\rho V \\ \rho U V \\ \rho V^2 + p \\ V \left( \rho E + p \right)
\end{pmatrix}
\end{equation}

With index notation:

\begin{equation}
\begin{array}{lcl}
\uu{R} &=& \epsilon \hat{\tau} \int \left( \dfrac{\partial^2}{\partial t^2} 
                                       + 2 u_i \dfrac{\partial^2}{\partial t \partial x_i}
                                       + u_i u_j \dfrac{\partial^2}{\partial x_i \partial x_j}
                                  \right) g \pdxi \\
       &+& \epsilon \int \left( \left( \ppt \hat{\tau} \right)
                                       \underbrace{\left( \ppt g + u_i \pp{x_i} g \right)
                                                }_{ = \mathcal{O}\left( \epsilon \right) }
                              + \left( u_i \pp{x_i} \hat{\tau} \right) \left( \ppt g + u_i \pp{x_i} g \right)
                         \right) \pdxi \\
       &=& \epsilon \hat{\tau} \int \ppt \left( \underbrace{\left( \ppt g + u_i \pp{x_i} g \right)
                                                }_{ = \mathcal{O}\left( \epsilon \right) }
                                          + u_i \pp{x_i} \left( \ppt g + u_i \pp{x_i} g
                                  \right) \right) \pdxi \\
       &+& \epsilon \int \left( u_i \pp{x_i} \hat{\tau} \left( \ppt g + u_i \pp{x_i} g \right) \right) \pdxi \\
\end{array}
\end{equation}

Inverse productrule:

\begin{equation}
\begin{array}{lcl}
\uu{R} &=& \epsilon \pp{x_i} \int u_i \hat{\tau} \left( \ppt g + u_i \pp{x_i} g \right) \pdxi \\
       &=& \epsilon \ppx 
\end{array}
\end{equation}



\end{document}